SCZ intro

\section{Overview}
Schizophrenia (SCZ) is a chronic, debilitating disease affecting approximately 1$\%$ of the population\citep{NIMHb}.
As the underlying aetiology is still poorly understood, SCZ is defined by its symptoms, the most prominent of which is psychosis.
Symptoms are generally classified into three main categories: positive, negative, and cognitive \citep{Kay1982}\citep{NIMHa}.
Positive symptoms are behaviors not normally seen in healthy individuals and include hallucinations, delusions, suspiciousness, hostility, thought disorders, and movement disorders.
Negative symptoms reflect missing or disrupted normal though processes, including: blunted affect, social withdrawal, avolition, anhedonia, and difficulty in abstract thinking.
Cognitive symptoms include attentional deficits, poor executive functioning, both working and long-term memory impairment, as well as spatial and episodic memory deficits.
While the positive symptoms, in particular hallucinations and delusions, might be the most prominent symptoms of SCZ, cognitive symptoms are present in up to 75$\%$ of SCZ patients and the severity of cognitive symptoms is strongly linked to functional outcomes \citep{O'Carroll2000}\citep{Green1996}\citep{Keefe2007}.



underlying cause unknown
this is why we need to understand pathophysiology

Since the underlying causes of SCZ are not known, treatment for patients with SCZ is limited to treating symptoms as they present.
Treatment for SCZ patients primarily includes a combination of antipsychotic medication and psychosocial therapy \citep{NIMHa}.
Patients’ responses to antipsychotics varies widely, some experiencing complete remission of psychoses and others showing no alleviation of symptoms.
Overall, following a first episode-psychosis, long-term antipsychotic treatment are effective in 30-40$\%$ of patients in preventing future psychoses \citep{Boyer2007}\citep{Insel2010}, leaving over half of all patients still experiencing psychotic events.
More importantly, antipsychotic medication only aim to treat the positive symptoms of SCZ, such as hallucinations and delusion.
Treatment options are extremely limited for the debilitating cognitive deficits – including deficits in attention, executive function, and both working and episodic memory, which have proven to be the greatest barrier to rehabilitation \citep{Lieberman2005}\citep{Harrison2001}\citep{O'Carroll2000}\citep{Hyman2003}.


\subsection{Anatomy}
Brain structural changes have been reliably identified in SCZ patients following a first episode of psychosis, and more recently, in high-risk individuals during the prodrome, prior to a clinical diagnosis.
In particular, reductions in gray matter volume and a corresponding increase in lateral ventral volume have been consistently demonstrated \citep{Fusar-Poli2013}\citep{Shepherd2012}.
As we become aware of susceptibility genes, environmental triggers, and other prodromal markers of SCZ, the identification of an at-risk pool has allowed for the tracking of brain changes prior to the onset of psychosis.
Tracking of clinically high-risk individuals has identified SCZ-related structural changes in the hippocampal formation, the cerebellum, the superior and medial temporal lobe, the insular cortex and the prefrontal cortex \citep{Cannon2015}\citep{Millan2016}.
Additionally, among the clinically high-risk population, continued progression towards SCZ can be predicted by monitoring gray-matter loss \citep{Tognin2014}, aiding in pre-symptomatic identification with the hope of early intervention treatments (see \nameref{sec:intro:SCZ:neurodevelopment}).

\subsection{Memory deficits}
Disruption in working and declarative memory are among the primary cognitive deficits observed in SCZ patients.
The memory deficits present in SCZ are specific in that patients seem to have relatively spared implicit or procedural memory, while many patients present with striking deficits in declarative memory, including both episodic memory (conscious recollection of events) and semantic memory (knowledge of people, place, objects, and facts) \citep{O'Carroll2000}\citep{Aleman1999}\citep{Gold2010}.
In addition to being specific, the observed memory deficits are also pervasive, in that they can not be accounted for by age, education, medication, disease duration or severity \citep{Ranganath2008} and are present in the majority of SCZ patients \citep{}.
As mentioned more generally about cognitive symptoms of SCZ, the severity of memory impairments in particular is one of the strongest predictors of longterm functional outcome among SCZ patients \citep{Green1996}, as failures in declarative memories provide the greatest barrier to steady employment.

The pattern of memory deficits observed in SCZ patients is consistent with disruptions in both the executive control and memory strategies associated with the prefrontal cortex (PFC) as well as the longterm memory storage function of the hippocampus (HPC).
Disorders that specifically arise from PFC lesions or dysfunction show a similar impairment in encoding and retrieval as seen in SCZ patients \citep{Ranganath2008}.
Patients do not make use of the same semantic memory strategies as healthy individuals, though these impairments can at least in part be ameliorated by training in semantic memory strategies or through restructuring of task stimuli (e.g. blocked instead of unblocked word lists) \citep{2 entries from tonight}.

\subsubsection{Spatial memory}
As described in more detail above (see \nameref{sec:intro:HPC:spatial}), memory for specifics locations is one particular type of episodic memory that has proven to be a tractable means for studying memory.


\subsection{Mechanistic hypotheses}
While the underlying cause of schizophrenia has remained elusive, collective evidence has pointed to several possible mechanistic pathways.
Narrowing down the disease etiology has been confounded by a high concordance with other psychiatric diseases.
Unlike Alzheimer’s disease or Parkinson’s disease where cell-death of specific cell types has been causally linked to disease progression, SCZ and many other psychiatric disorders exhibit no drastic cell loss or increased gliosis, and affected brain regions can span both cortical and subcortical regions, suggesting network or circuit dysfunctions \citep{Uhlhaas2012}\citep{Lewis2002}.
\subsubsection{Dopamine}
The earliest mechanistic hypotheses for the disease progression of SCZ involves disruption in normal dopamine signaling \citep{Matthysse1973}.
This hypothesis revolves around the positive symptoms of SCZ, specifically the hallucinations and delusions which can be both induced and prevented through modification of dopamine levels in the brain.
In particular, compounds that generally increase the level of dopamine (e.g. amphetamine, LSD) lead to hallucinations similar to those experienced by SCZ patients \citep{Angrist1994}\citep{Lieberman1987}.
Conversely, for over 60 years the primary treatment for SCZ has been antipsychotic medications \citep{Delay1952}, which were first shown to increase the metabolism of dopamine \citep{Carlsson1963} and are now known to function primarily as D2 dopamine receptor antagonists \citep{}.
The general effectiveness of classical antipsychotics (e.g. haloperidol) and the newer atypical antipsychotics (e.g. clozapine) in treating the positive symptoms of SCZ, suggested a central role for dopamine in the more general neuropatholysiology of SCZ, but this insight has failed to expand beyond the direct treatment of psychotic episodes.
Indeed, while D2 dopamine receptor antagonists work well for many patients in managing psychotic episodes, these treatments have done very little in improving functional outcomes, as the more debilitating negative and cognitive symptoms remain untreated \citep{Insel2010}.

The role of dopamine in SCZ is now realized to be more nuanced than a simple consequence of a brain-wide hyperdopaminergic state.
While D2 dopamine receptor antagonist antipsychotic treatments do help manage psychotic episodes in many SCZ patients, in some these treatments show no effect.
Additionally, post-mortem measurements of dopamine levels in SCZ patients, as well as \emph{in vivo} measurements of dopamine activity, have not been entirely consistent across studies, and more importantly, across patients \citep{}.
This inconsistency is most likely reflective of variability in underly disease etiology, and points to a lack of a singular disease pathology, but rather a collection of underlying brain dysfunctions that funnel towards shared pathways and a commonly identified set of expressed symptoms.

Using radiolabeled L-dopa, presynaptic dopamine levels have been consistently shown to be elevated in the striatum of SCZ patients \citep[for review, see][]{Howes2007}
At the same time, also in the striatum, D2/3 dopamine receptors (the primary receptor subtypes in this brain region) are modestly (10-20$\%$) elevated, independent of treatment with antipsychotic drugs \citep[for review, see][]{Howes2009}.
In contrast to the striatum, D1 dopamine receptors predominantly mediate dopamine activity, and while the effect is not as consistent, receptor levels show an association with severity of cognitive and negative symptoms in SCZ \citep{Goldman-Rakic2004}.


D2r presynaptic striatum

hypodopamine frontal?

dopamine spcific to psychoses

many underlying causes that converge on dopamine (complex genetic interactions and environmental hits)

subcortical hyperdopaminergia, prefrontal hypodopaminergia

\subsubsection{Glutamate/GABA}
\subsection{Schizophrenia as a neurodevelopmental disorder}\label{sec:intro:SCZ:neurodevelopment}

The typical age for a first psychotic episode is 18-25 years, though there is 

\section{Genetics}
Significant recent advances in understanding SCZ have come from identifying a large network of genetic abnormalities that predispose for SCZ.
\subsection{22q11.2}
22q11.2
\subsubsection{Df16}
Df16