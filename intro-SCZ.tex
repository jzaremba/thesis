
\section{Overview}
\Scz/ is a chronic, debilitating disease affecting approximately 1$\%$ of the population\citep{NIMHb}.
As the underlying aetiology is still poorly understood, \scz/ is defined by its symptoms, the most prominent of which is psychosis.
Symptoms are generally classified into three main categories: positive, negative, and cognitive \citep{Kay1982, NIMHa}.
Positive symptoms are behaviors not normally seen in healthy individuals and include hallucinations, delusions, suspiciousness, hostility, thought disorders, and movement disorders.
Negative symptoms reflect missing or disrupted normal thought processes, including: blunted affect, social withdrawal, avolition, anhedonia, and difficulty in abstract thinking.
Cognitive symptoms include attentional deficits, poor executive functioning, both working and long-term memory impairment, specifically in the domains of episodic semantic memory deficits.
While the positive symptoms, in particular hallucinations and delusions, might be the most prominent symptoms of \scz/, cognitive symptoms are present in up to 75$\%$ of \scz/ patients and the severity of cognitive symptoms is strongly linked to functional outcomes \citep{O'Carroll2000, Green1996, Keefe2007}.
This is particularly troubling since treatment options for the negative and cognitive symptoms of \scz/ are extremely limited.

Since the underlying causes of \scz/ are not known, treatment for patients with \scz/ is limited to treating symptoms as they present.
Treatment for \scz/ patients primarily includes a combination of antipsychotic medication and psychosocial therapy \citep{NIMHa}.
Patients’ responses to antipsychotics varies widely, some experiencing complete remission of psychoses and others showing no alleviation of symptoms.
Overall, following a first episode-psychosis, long-term antipsychotic treatment are effective in 30-40$\%$ of patients in preventing future psychoses \citep{Boyer2007, Insel2010}, leaving over half of all patients still experiencing psychotic events.
More importantly, antipsychotic medication only aim to treat the positive symptoms of \scz/, such as hallucinations and delusions.
Treatment options are extremely limited for the debilitating cognitive deficits - including deficits in attention, executive function, and both working and episodic memory, which have proven to be the greatest barrier to rehabilitation \citep{Lieberman2005, Harrison2001, O'Carroll2000, Hyman2003}.

In the following sections I will expand upon some of the symptoms, causes, and brain abnormalities observed in \scz/ patients, with a specific focus on aspects of the disease most relevant to my main research project: \autoref{ch:df}.
I will discuss details of the cognitive dysfunctions present in \scz/ patients [\ref{sec:intro:scz:memory}], with a particular focus on spatial and episodic memory impairments.
I will present what we do know about underlying genetic and environmental causes of \scz/ [\ref{sec:intro:scz:genes_and_environment}], including details of 22q11.2 deletion syndrome and a mouse model of this deletion, \df/.
Finally, I will lay out several proposed proximate causes of \scz/ symptoms including brain anatomical changes, alterations in neumodulatory signaling, and an imbalance in excitation and inhibition [\ref{sec:intro:scz:causes}].


\section{Memory deficits}\label{sec:intro:scz:memory}
Disruption in working and declarative memory are among the primary cognitive deficits observed in \scz/ patients.
The memory deficits present in \scz/ are specific in that patients seem to have relatively spared implicit or procedural memory, while many patients present with striking deficits in declarative memory, including both episodic memory (conscious recollection of events) and semantic memory (knowledge of people, place, objects, and facts) \citep{O'Carroll2000, Aleman1999, Gold2010}.
In addition to being specific, the observed memory deficits are also pervasive, in that they can not be accounted for by age, education, medication, disease duration or severity \citep{Ranganath2008} and are present in the majority of \scz/ patients \citep{}.
As mentioned more generally about cognitive symptoms of \scz/, the severity of memory impairments in particular is one of the strongest predictors of longterm functional outcome among \scz/ patients \citep{Green1996}, as failures in declarative memories provide the greatest barrier to steady employment.

The pattern of memory deficits observed in \scz/ patients is consistent with disruptions in both the executive control and memory strategies associated with the prefrontal cortex (PFC) as well as with the longterm memory storage function of the hippocampus (HPC).
Disorders that specifically arise from PFC lesions or dysfunction show a similar impairment in encoding and retrieval as seen in \scz/ patients \citep{Ranganath2008}.
Patients do not make use of the same semantic memory strategies as healthy individuals, though these impairments can at least in part be ameliorated by training in semantic memory strategies or through restructuring of task stimuli (e.g. blocked instead of unblocked word lists) \citep{Gold1992, Stone1998}.

\subsection{Spatial memory}
As described in more detail above (see \nameref{sec:intro:HPC:spatial}), memory for specifics locations is one particular type of episodic memory that has proven to be a tractable means for studying memory.
Spatial memory deficits are well documented in \scz/ patients \citep{Boyer2007, Hanlon2006, Aleman1999}, but the underlying circuit level dysfunctions have rarely been investigated \citep{Hayashi2015, Suh2013}.


\section{Genes and the Environment}\label{sec:intro:scz:genes_and_environment}
Significant recent advances in understanding \scz/ have come from identifying a large network of genetic abnormalities that predispose for \scz/.

\subsection{Schizophrenia as a neurodevelopmental disorder}\label{sec:intro:scz:neurodevelopment}

The typical age for a first psychotic episode is 18-25 years, though there is 

\subsection{22q11.2}
Evidence suggests that \scz/ has up to 80$\%$ heritability, and in particular, genetic studies have unequivocally implicated a region of human chromosome 22 (22q11.2) in disease risk \citep{Karayiorgou1995, Chow2006, Karayiorgou2010}.
Individuals with 22q11.2 microdeletions exhibit a spectrum of cognitive deficits as children, and $~$30$\%$ of them develop \scz/ in adolescence or early adulthood, accounting for up to 2$\%$ of all \scz/ cases \citep{Karayiorgou2010}.

\subsubsection{Df16}
Advances in understanding the genetic component of \scz/ predisposition have led to the development of etiologically-validated mouse models of \scz/ which will allow for the direct analysis of \emph{in vivo} neuronal network activity and identification of circuit dysfunctions present in the disease.
In order to dissect altered hippocampal circuitry in \scz/ I will be using the etiologically validated mouse model of the 22q11.2 microdeletion (\df/), generated by deleting the syntenic region in the mouse chromosome which contains most of the same 28 genes deleted in humans \citep{Stark2008}.
Behavioral tests on the \df/ mouse model have shown increased non-specific anxiety in open field and light-dark transition tasks, impaired sensorimotor gating by measuring startle response during prepulse inhibition (PPI), impaired spatial working memory in a delayed non-matching to place T-maze task, impaired working memory characterized by altered hippocampal-prefrontal synchrony, and decreased freezing in a contextual fear conditioning paradigm, which could reflect either impaired contextual memory or the inability to associate the unconditioned stimuli with the context in which it was presented \citep{Drew2011b, Stark2008, Sigurdsson2010}.
Additionally, a study of immediate early gene expression using c-Fos staining following exposure to a novel environment found significantly fewer CA3 pyramidal cells active and a trend in the same direction for CA1 pyramidal cells as well, which is suggestive of deficits in spatial memory \citep{Drew2011b}.
Taken together, the \df/ mouse model of the human \scz/-linked 22q11.2 microdeletion effectively recapitulates cognitive deficits seen in \scz/ patients, including increased anxiety, impaired executive control, and deficits in spatial and episodic memory.


\section{Mechanistic hypotheses}\label{sec:intro:scz:causes}
While the underlying cause of schizophrenia has remained elusive, collective evidence has pointed to several possible mechanistic pathways.
Narrowing down the disease etiology has been confounded by a high concordance with other psychiatric diseases.
Unlike Alzheimer’s disease or Parkinson’s disease where cell-death of specific cell types has been causally linked to disease progression, \scz/ and many other psychiatric disorders exhibit no drastic cell loss or increased gliosis, and affected brain regions can span both cortical and subcortical regions, suggesting network or circuit dysfunctions \citep{Uhlhaas2012, Lewis2002}.

\subsection{Anatomy}
Brain structural changes have been reliably identified in \scz/ patients following a first episode of psychosis, and more recently, in high-risk individuals during the prodrome, prior to a clinical diagnosis.
In particular, reductions in gray matter volume and a corresponding increase in lateral ventral volume have been consistently demonstrated \citep{Fusar-Poli2013, Shepherd2012}.
As we become aware of susceptibility genes, environmental triggers, and other prodromal markers of \scz/, the identification of an at-risk pool has allowed for the tracking of brain changes prior to the onset of psychosis.
Tracking of clinically high-risk individuals has identified \scz/-related structural changes in the hippocampal formation, the cerebellum, the superior and medial temporal lobe, the insular cortex and the prefrontal cortex \citep{Cannon2015, Millan2016}.
Additionally, among the clinically high-risk population, continued progression towards \scz/ can be predicted by monitoring gray-matter loss \citep{Tognin2014}, aiding in pre-symptomatic identification with the hope of early intervention treatments (see \nameref{sec:intro:scz:neurodevelopment}).

\subsubsection{Hippocampus}
Consistent behavioral, anatomical, and physiological studies of \scz/ patients have all pointed towards the HPC as an integral region in disease pathology \citep{Boyer2007, Bogerts1985, Jakob1986}.
Of these subcortical projections the medial septum projections are of particular interest, since the cholinergic projection has been linked to learning and memory \citep{Parent2004} and the GABAergic projection is known to target PV+ interneurons \citep{Freund1988} which are altered in \scz/ (see \nameref{sec:intro:scz:glutamate}) and it has also been linked to aberrant behavior in an NMDAR antagonist model of schizophrenia \citep{Ma2012}.
Despite these well-characterized features of hippocampal organization, it remains unknown how the function of these circuit motifs are altered in \scz/ during hippocampal-dependent behaviors.

\subsection{Dopamine}
The earliest mechanistic hypotheses for the disease progression of \scz/ involves disruption in normal dopamine signaling \citep{Matthysse1973}.
This hypothesis revolves around the positive symptoms of \scz/, specifically the hallucinations and delusions which can be both induced and prevented through modification of dopamine levels in the brain.
In particular, compounds that generally increase the level of dopamine (e.g. amphetamine, LSD) lead to hallucinations similar to those experienced by \scz/ patients \citep{Angrist1994, Lieberman1987}.
Conversely, for over 60 years the primary treatment for \scz/ has been antipsychotic medications \citep{Delay1952}, which were first shown to increase the metabolism of dopamine \citep{Carlsson1963} and are now known to function primarily as D2 dopamine receptor antagonists \citep{}.
The general effectiveness of classical antipsychotics (e.g. haloperidol) and the newer atypical antipsychotics (e.g. clozapine) in treating the positive symptoms of \scz/, suggested a central role for dopamine in the more general neuropatholysiology of \scz/, but this insight has failed to expand beyond the direct treatment of psychotic episodes.
Indeed, while D2 dopamine receptor antagonists work well for many patients in managing psychotic episodes, these treatments have done very little in improving functional outcomes, as the more debilitating negative and cognitive symptoms remain untreated \citep{Insel2010}.

The role of dopamine in \scz/ is now realized to be more nuanced than a simple consequence of a brain-wide hyperdopaminergic state.
While D2 dopamine receptor antagonist antipsychotic treatments do help manage psychotic episodes in many \scz/ patients, in some these treatments show no effect.
Additionally, post-mortem measurements of dopamine levels in \scz/ patients, as well as \emph{in vivo} measurements of dopamine activity, have not been entirely consistent across studies, and more importantly, across patients \citep{}.
This inconsistency is most likely reflective of variability in underly disease etiology, and points to a lack of a singular disease pathology, but rather a collection of underlying brain dysfunctions that funnel towards shared pathways and a commonly identified set of expressed symptoms.

Using radiolabeled L-dopa, presynaptic dopamine levels have been consistently shown to be elevated in the striatum of \scz/ patients \citep[for review, see][]{Howes2007}
At the same time, also in the striatum, D2/3 dopamine receptors (the primary receptor subtypes in this brain region) are modestly (10-20$\%$) elevated, independent of treatment with antipsychotic drugs \citep[for review, see][]{Howes2009}.
In contrast to the striatum, D1 dopamine receptors predominantly mediate dopamine activity, and while the effect is not as consistent, receptor levels show an association with severity of cognitive and negative symptoms in \scz/ \citep{Goldman-Rakic2004}.


D2r presynaptic striatum

hypodopamine frontal?

dopamine spcific to psychoses

many underlying causes that converge on dopamine (complex genetic interactions and environmental hits)

subcortical hyperdopaminergia, prefrontal hypodopaminergia

\subsection{Glutamate/GABA}\label{sec:intro:scz:glutamate}
It has been proposed that an imbalance in levels of excitatory and inhibitory activity during development underlies \scz/ \citep{Insel2010, Coyle2006, Yizhar2011}.
Support for this theory includes the observation that NMDA receptor (NMDAR) antagonists, such as phencyclidine and ketamine, induce psychoses in healthy subjects \citep{Morris2005}, worsen symptoms in \scz/ patients, and cause schizophrenia-like symptoms in mice \citep{Inta2010}, while agonists for the NMDAR glycine binding site show potential to alleviate symptoms \citep{Tsai1998}.
Additionally, post-mortem studies have found decreased levels of pre-synaptic GABAergic machinery, including GAT, GAD67, and parvalbumin (PV), in both the prefrontal cortex (PFC) and HPC \citep{Coyle2006, Zhang2002, Konradi2011}.
In particular, \scz/ patient have shown decreased levels of parvalbumin leading to hypofunction of PV+ interneurons in the prefrontal cortex as well as decreased numbers of PV+ interneurons in the HPC \citep{Zhang2002, Lewis2005} and parvalbumin interneurons are required for normal spatial working memory functions \citep{Korotkova2010, Murray2011}.
However, there is currently limited data available on how activity of hippocampal GABAergic interneurons is altered in \scz/.
