SCZ intro

\section{Overview}
Schizophrenia (SCZ) is a chronic, debilitating disease affecting $\approx$1$\%$ of the population in the US \citep{NIMHb}.
Symptoms are generally classified into three main categories: positive, negative, and cognitive \citep{Kay1982}\citep{NIMHa}.
Positive symptoms are behaviors not normally seen in healthy individuals and include hallucinations, delusions, suspiciousness, hostility, thought disorders, and movement disorders.
Negative symptoms reflect missing or disrupted normal though processes, including: blunted affect, social withdrawl, avolition, anhedonia, and difficulty in abstract thinking.
Cognitive symptoms include attentional deficits, poor executive functioning, both working and long-term memory impairment, as well as spatial and episodic memory deficits.
While the positive symptoms, in particular hallucinations and delusions, might be the most prominent symptoms of SCZ, cognitive symptoms are present in up to 75$\%$ of SCZ patients and the severity of cognitive symptoms is strongly linked to functional outcomes \citep{O'Carroll2000}\citep{Green1996}\citep{Keefe2007}.
The cognitive deficits present in SCZ are specific in that patients seem to have relatively spared implicit or procedural memory, while many patients present with striking deficits in declarative memory, including both episodic memory (conscious recollection of events) and
semantic memory (knowledge of people, place, objects, and facts) \citep{4,7–9}.


underlying cause unknown
this is why we need to understand patho-physiology

Since the underlying causes of SCZ are not known, treatment for patients with SCZ is limited to treating symptoms as they present.
Treatment for SCZ patients primarily includes a combination of antipsychotic medication and psychosocial treatments \citep{NIMHa}.
Patient’s response to antipsychotics varies widely, some experiencing complete remission of psychoses and others showing no alleviation of symptoms.
Overall, following a first episode-psychosis, long-term antipsychotic treatment are effective in 30-40$\%$ of patients in preventing future psychoses \citep{Boyer2007}\citep{Insel2010}, leaving over half of all patients still experiencing psychotic events, and more importantly, no treatment for the debilitating cognitive deficits – including deficits in attention, executive function, and both working that have proven to be the greatest barrier to rehabilitation \citep{Lieberman2005}\citep{Harrison2001}\citep{O'Carroll2000}.

\subsection{Anatomy}

\subsection{Oscilations?}

\subsection{Spatial learning deficits}
Spatial learning deficits

\subsection{Mechanistic hypotheses}
\subsubsection{Dopamine}
\subsubsection{Glutamate}
\subsubsection{GABA}
\subsubsection{Neurodevelopment}

\section{22q11.2}
22q11.2
\subsection{Df16}
Df16