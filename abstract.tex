\acresetall
Recognizing and understanding where and when events occurred is essential for normal learning and memory of life experiences.
Disruptions in the normal processing of spatial and episodic memories can have devastating consequences; in particular this is one component of the debilitating cognitive deficits of schizophrenia (SCZ).
% By improving our understanding of the way in which the brain process, encodes, and recalls episodic memory, we can gain a better foundation for 
We are just now beginning to understand the molecular changes in SCZ, but still very little is known about how neural circuit are disrupted and lead to behavioral and cognitive dysfunction.
In my thesis I will attempt to address two primary questions; how does hippocampal circuitry support spatial-episodic memories, and what goes wrong when these memories are impaired?

First, how precisely do hippocampal circuits support spatial and episodic learning?
% This is not a particularly novel question, and in many ways we already know the answer.
In 1885 Hermann Ebbinghaus published the first results of a quantitative study of the psychology of memory, showing the predictable forgetting of items over time. We eventually realized that memory is not a unitary function of the brain, but that it is dissociable at it's broadest level in to explicit and implicit memory functions.
We have now identified large regions of the brain that are essential for these functions.
The first functional imaging of the human brain further advanced out understanding of the particular brain regions active during memory tasks and technological advances have allowed us to generate higher resolution functional maps of the brain. Moving in to rodent models, we are now getting closer to the \textsc{memory engram}, the set of changes that occur in the brain that stored an object, event, or association for future recall. In some particular instances, such as spatial and episodic memories, we already have a very good understanding. But, how does that memory come to be? Which particular cells store this information? In my primary thesis project, I will show that principal cells in hippocampal area CA1 stabilize their activity as mice learn a spatial reward learning task, and more specifically, the way in which the population of cells shift their firing activity to encode the reward location.

Second, what goes wrong in the normal processing of information that leads to disrupted memory storage and recall?
Deficits in spatial and episodic memory are two of the primary cognitive dysfunctions in SCZ; cognitive functions that have been linked unequivocally to the hippocampus (HPC).
Cognitive deficits observed in \scz/ patients are at their core, neuronal circuit disruptions.
What can we learn about the circuits underlying these behavioral symptoms?
What goes wrong in the brain that is driving these disruptions?
By recording the activity of CA1 place cells in an etiologically-validated mouse model of \scz/ while the mice are actively engaged in a spatial learning task I identified specific features of the place cell population that predict behavioral deficits.

Taken together, my work used head-fixed two-photon function imaging of awake schizophrenia-mutant and wildtype mice in order to directly probe hippocampal circuits involved in spatial learning to identify the underlying cellular substrates of both healthy and diseased spatial memory processing.

% In the following sections I will lay out what we know about memory formation and consolidation, in particular \ac{HPC}.


% Our goal is to investigate neural circuit dysfunctions in SCZ in order to
% better understand the severe cognitive deficits underlying the disease. 
% This proposal uses recent advances in neuroscience tools to probe neural circuits in ways not
% previously possible. It will make use of recent progress in understanding the genetic factors leading to
% schizophrenia by studying the etiologically-validated Df(16)A deletion mouse model of the human 22q11.2
% deletion syndrome mutation which is highly associated with schizophrenia. This project will test the hypothesis
% that hippocampal area CA1 principal neurons are disrupted during spatial navigation and goal-directed spatial
% learning in the schizophrenia mouse model by using two-photon imaging of genetically-encoded calcium-
% indicators in awake head-fixed mice to record the response of populations of hundreds of pyramidal cells
% during these behaviors. Preliminary data suggests that spatial representations are more rigid, they don’t as
% readily adapt over time. It is known that spatial memory can be robustly modulated by novelty, salience, and
% attention, all of which are signaled by neuromodulatory inputs to the hippocampus. The proposed research will
% test the hypothesis that disrupted stability in a schizophrenia mouse model is caused by altered dopaminergic
% or cholinergic activity, by using a combination of awake in vivo activity imaging and optogenetic manipulations.
% Taken together, this proposed research will use head-fixed two-photon function imaging of awake
% schizophrenia-mutant mice in order to directly probe hippocampal circuit dysfunctions linked to cognitive
% deficits in schizophrenia, potentially elucidating novel targets for treatment.



% Formulate questions here that will be further introduced in intro.

% Place cells are basis of spatial memory system, but what happens during learning?

% Deficits in SCZ, what do place cells look like when performance is impaired?

% Can we identify a functional correlate of impaired spatial learning?

% Collectively further understanding of all pieces of hippocampal memory system: long-range context-projecting interneurons, new born GC, develpomentally distinct subpopulations in CA1.
