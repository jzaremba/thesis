Learning and memory define who we are and shape who we will become.
Memory allows us to adapt to the world.
Memory and it's various components -- learning, adaptation, plasticity, association -- are fundamental features of nervous system across all organisms.

\section{Structure}
The hippocampus i

The principal neurons in the hippocampal formation communicate through the classically-defined trisynaptic loop: perforant path fibers project from the entorhinal cortex (EC) to granule cells in the dentate gyrus, which in turn send mossy fiber projections to CA3 that finally travel along the Schaffer collateral pathway and synapse upon proximal apical and basal dendrites of CA1 pyramidal cells (CA1PCs).
CA1PCs then send processed information to cortical and subcortical areas, including return projections back to deep layers of EC.
In addition, distal dendrites of CA1PCs receive direct projections from EC along the temporammonic pathway and CA1PC activity is also directly regulated by a diverse population of local GABAergic interneurons.
Finally, both principal neurons and GABAergic interneurons are targeted by afferents from neuromodulatory nuclei, including cholinergic and GABAergic projections from the medial septum  \citep{Klausberger2008}, serotonergic and glutamatergic projections from the raphe nuclei \citep{Varga2009}, as well as dopaminergic and noradrenergic projections from the ventral tegmental area \citep{Gasbarri1997} and locus coeruleus \citep{Foote1983}.


Burwell RD (2000): The parahippocampal region: Corticocortical connectivity. Ann N Y Acad Sci 911:25–42.
\section{Memory}




The hippocampus has been central to our study of memory since Scoville and Milner first reported in the 1950's on patient H.M. who had profound anterograde amnesia following the bilateral removal of large portions of the medial temporal lobes, which includes the HPC and parahippocampal structures \citep{Scoville1957}.
Since then, numerous studies have shown that the HPC is essential for normal formation and recall of long term episodic and semantic memory \citep[reviewd in][]{Eichenbaum2000, Burgess2002}.

Episodic memory is 
Episodic memory \citep{Tulving1972}
% Ebbinghaus H (1885). Über das Gedachtnis, Dunker & Humbolt.
% https://books.google.com/books?id=kfA0AAAAMAAJ&ots=hjMhksDE3X&dq=Ebbinghaus%20H%20(1885).%20%C3%9Cber%20das%20Gedachtnis%2C&lr&pg=PA1#v=onepage&q&f=false

% \begin{quote}
% Episodic memory retrieves and stores information about temporally dated episodes or events, and temporal-spatial relations among these events. A perceptual event can be stored in the episodic system solely in terms of its perceptible properties or attributes, and it is always stored in terms of its autobiographical reference to the already existing contents of the episodic memory store. The act of retrieval of information from the episodic memory store, in addition to making the retrieved contents accessible to inspection, also serves as a special type input into episodic memory and thus changes the contents of the episodic memory store. The system is probably quite susceptible to transformation and loss of information. While the specific form in which perceptual input is registered into the episodic memory can at times be strongly influenced by information in semantic memory-we refer to the phenomenon as encoding-it is also possible for the episodic system to operate relatively independently of the semantic system.
% \attrib{\citealt[pgs.~385-386]{Tulving1972}}
% \end{quote}

% \begin{quote}
% Consider now a typical memory experiment in which a subject is asked to study and remember a list of familiar words or pair of words. This is an episodic memory task. The occurrence of a verbal item in a given list, at a particular time, and in specific temporal relation to other items in the list is an autobiographical episode having no necessary extra-episodic denotative reference. The subject has successfully retrieved information about this episode when he responds to the retrieval query with the reproduction if an appropriate copy of the input item.
% \attrib{\citealt{Tulving1972}}
% \end{quote}

% \begin{quote}
% Each experienced event always occurs at a particular spatial location and in a particular temporal relation to other events that already have occurred, events occurring simultaneously with it, or events that have not yet occurred. These temporal relations among experienced events are also somehow represented as properties of items in the episodic memory system. To ask a person about some item in episodic memory means to ask them when did event $E$ happen, or what events happened at time $T$. Retrieval of information of this kind from episodic memory is successful if the person can describe the perceptible properties of the event in question and more or less accurately specify its temporal relations to other events. Temporal coordinates of an event and its representation in episodic memory of course need not be specified in terms of the clock and the calendar. They could be recorded in terms of temporal occurrences of other events in some as yet little understood manner.
% \attrib{\citealt[pg.~388]{Tulving1972}}
% \end{quote}

Eichenbaum H, Yonelinas AR, Ranganath C (2007): The medial tempo- ral lobe and recognition memory. Annu Rev Neurosci 30:123–152.
Milner B, Squire LR, Kandel ER (1998): Cognitive neuroscience and the study of memory. Neuron 20:445–468.
23.
Fortin NJ, Wright SP, Eichenbaum H (2004): Recollection-like memory retrieval in rats is dependent on the hippocampus. Nature 431:188– 191.
Sauvage MM, Fortin NJ, Owens CB, Yonelinas AP, Eichenbaum H (2008): Recognition memory: Opposite effects of hippocampal dam- age on recollection and familiarity. Nat Neurosci 11:16–18.

\subsection{Spatial memory}\label{sec:intro:hpc:spatial}
Studied because it is tractable.
We have good behavioral tests in rodents.
We found cells that appear to encode the memory itself.
Allocentric navigation in particular is HPC-dependent \citep{OKeefe1978, Smith1989}, not egocentric.

\subsubsection{Spatial memory is a core component of episodic memory}\label{sec:intro:hpc:spatial-episodic}
In 1972 Endel Tulving coined the term ``episodic memory'', describing it as follows:

\begin{quote}
Episodic memory retrieves and stores information about temporally dated episodes or events, and temporal-spatial relations among these events...Each experienced event always occurs at a particular spatial location and in a particular temporal relation to other events that already have occurred, events occurring simultaneously with it, or events that have not yet occurred.
\attrib{\citealt{Tulving1972}}
\end{quote}

As this interpretation suggests, the specific of the events itself are inseparable from the time at which it occurred and the location of the event (or individual aspects of the event).
This is most obvious in autobiographical episodic memory, where experiences (e.g. waiting in line to get lunch) are remembered along with the location (e.g. Mike's Bagels at 168th and Broadway) and the time (e.g. last Tuesday around 2PM) they occurred.
This applies to psychological memory tests as well, such as remembering words on a list, where the temporal order of items on the list (I saw ``orange'' before ``banana'') and the visio-spatial arrangement of the words on the paper list (``boat'' was written above ``car'') are core components of the stored memory.


\section{Place cells}
Pyramidal cells in hippocampal area CA1 are the principal excitatory neuron in that region and the primary output from the hippocampus.
As an animal explores an environment these pyramidal cells show sparse spatially-modulated changes in firing rates (place cells) that are established rapidly and subsequently remain stable \citep{O'Keefe1971}\citep{Frank2004}.
These place cells form an allocentric map of the environment, which is essential for normal episodic memory function \citep{Smith2006c}\citep{Nakazawa2004}\citep{Buzsaki2013}.


\section{Spatial learning}
Spatial learning

\section{Reward learning}