HPC into

\section{Overview}
Overview

\section{Structure}
Burwell RD (2000): The parahippocampal region: Corticocortical connectivity. Ann N Y Acad Sci 911:25–42.

The principal neurons in the hippocampal formation communicate through the classically-defined trisynaptic loop: perforant path fibers project from the entorhinal cortex (EC) to granule cells in the dentate gyrus, which in turn send mossy fiber projections to CA3 that finally travel along the Schaffer collateral pathway and synapse upon proximal apical and basal dendrites of CA1 pyramidal cells (CA1PCs).
CA1PCs then send processed information to cortical and subcortical areas, including return projections back to deep layers of EC.
In addition, distal dendrites of CA1PCs receive direct projections from EC along the temporammonic pathway and CA1PC activity is also directly regulated by a diverse population of local GABAergic interneurons.
Finally, both principal neurons and GABAergic interneurons are targeted by afferents from neuromodulatory nuclei, including cholinergic and GABAergic projections from the medial septum  \citep{Klausberger2008}, serotonergic and glutamatergic projections from the raphe nuclei \citep{Varga2009}, as well as dopaminergic and noradrenergic projections from the ventral tegmental area \citep{Gasbarri1997} and locus coeruleus \citep{Foote1983}.

\section{Memory}
The hippocampus has been central to our study of memory since Scoville and Milner first reported in the 1950's on patient H.M. who had profound anterograde amnesia following the bilateral removal of large portions of the medial temporal lobes, which includes the HPC and parahippocampal structures \citep{Scoville1957}.
Since then, numerous studies have shown that the HPC is essential for normal formation and recall of long term episodic and semantic memory \citep[reviewd in][]{Eichenbaum2000}\citep{Burgess2002}.


Eichenbaum H, Yonelinas AR, Ranganath C (2007): The medial tempo- ral lobe and recognition memory. Annu Rev Neurosci 30:123–152.
Milner B, Squire LR, Kandel ER (1998): Cognitive neuroscience and the study of memory. Neuron 20:445–468.
23.
Fortin NJ, Wright SP, Eichenbaum H (2004): Recollection-like memory retrieval in rats is dependent on the hippocampus. Nature 431:188– 191.
Sauvage MM, Fortin NJ, Owens CB, Yonelinas AP, Eichenbaum H (2008): Recognition memory: Opposite effects of hippocampal dam- age on recollection and familiarity. Nat Neurosci 11:16–18.

\subsection{Spatial memory}\label{sec:intro:HPC:spatial}
Studied because it is tractable.
We have good behavioral tests in rodents.
We found cells that appear to encode the memory itself.

\section{Place cells}
Pyramidal cells in hippocampal area CA1 are the principal excitatory neuron in that region and the primary output from the hippocampus.
As an animal explores an environment these pyramidal cells show sparse spatially-modulated changes in firing rates (place cells) that are established rapidly and subsequently remain stable \citep{O'Keefe1971}\citep{Frank2004}.
These place cells form an allocentric map of the environment, which is essential for normal episodic memory function \citep{Smith2006c}\citep{Nakazawa2004}\citep{Buzsaki2013}.


\section{Spatial learning}
Spatial learning

\section{Reward learning}