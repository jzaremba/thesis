Techniques

\section{\emph{In vivo} two-photon calcium imaging}

\section{Head-fixed behavior}\label{sec:intro:techniques:behavior}
In order to allow for two-photon calcium imaging in awake mice, we need to design all behavioral tasks to work while mice are head-fixed under a two-photon microscope.
While the core apparatus was in place when I began experiments, extensive upgrades were necessary to optimally run my head-fixed learning tasks.
I, along with colleagues in the lab, redesigned the wheels, axles, and platform to minimize friction and facilitate running while head-fixed.
We designed a RFID-tag and quadrature encoded-based system to accurately track position over many laps to within $<$0.5~cm.
We designed new treadmill belts that consist of multiple fabrics sewed together with various cues scattered throughout, which together convey all the spatial information used by mice to anchor place cell maps.
I added features to our in-house behavioral control software to allow for the presentation
These hardware changes allowed for the design of new head-fixed behaviors, described below.

\subsection{Random foraging}
The first task we designed was a simple random foraging task where mice run to get water, but don't need to learn about any particular location on the belt in order to get water rewards.
All head-fixed paradigms begin with 1-2 weeks of training where mice learn to lick for water from a lick port while needing to run progressively farther in order to receive another reward (see \autoref{sec:df:methods:training}).
Mice were either operantly rewarded (they must lick to receive water) at random intervals along the belt or non-operantly at a fixed location.
This allowed me to look at baseline stability of spatial maps with two-photon calcium imaging over days or weeks (see \autoref{sec:applications:chronic}).

\subsection{Goal-oriented learning}\label{sec:intro:techniques:GOL}
In order to study spatial learning and memory while simultaneously imaging functional activity of hippocampal area CA1 place cells, I developed a head-fixed goal-oriented learning (GOL) task.
Freely-moving assays of spatial memory in rodents traditionally include the Morris water maze, Barnes maze, or cheese board maze (see \autoref{sec:intro:hpc:spatial-reward}).
While head-fixed under a two-photon microscope, mice are constrained to run in effectively a one-dimensional environment (similar to freely moving linear track paradigms) and there are limited options to experimentally assay choices made by the mice.
They key features that we were trying to design in the task were:
\begin{enumerate}
	\item Mice are able to learn a specific rewarded location on the treadmill that is otherwise un-cued.\label{item:into:techniques:GOL:location}
	\item Mice learn the task over several trials to allow for determining a `learning curve'.\label{item:into:techniques:GOL:learning}
	\item A behavioral readout that is sensitive to subtle variations in ability to learn.\label{item:into:techniques:GOL:readout}
	\item Be able to flexibly manipulate the reward and environment parameters to probe mice' ability to learn the task.\label{item:into:techniques:GOL:manip}
\end{enumerate}

Our GOL task requires mice to run laps along our multi-fabric, feature-rich circular treadmill belt.
The mice are water-restricted, so we use water delivered through a lick port as a reward.
Each lap there is one spatial region (\textsc{reward zone}, usually a 20-cm window) on the belt where the mice can operantly receive water rewards: if they lick they get water, if they do not lick, they don't.
The reward will `dry up' after after a fixed amount of time has passed since the mouse entered the reward zone that lap (usually 3 seconds).
Not all mice will run for water rewards at all, and not all mice will perform this task, but many are able to (\ref{item:into:techniques:GOL:location}).
We generally give the mice 9 sessions to learn the reward location.
Some mice will find the reward immediately, but even those that do, they continue to imptove and stabilize their performance over the 9 sessions (\ref{item:into:techniques:GOL:learning}).
By measuring the capacitance of the lick port, we can detect changes when the mouse licks, providing us an accurate measure of the time (and the mouse's location on the belt) when each lick occurs.
The operant nature of the reward schedule requires the mice to lick everywhere along the belt to sample each location and find the correct position that will be rewarded.
As the mice learn the reward zone, they suppress licking away from the reward and will then only lick at the reward zone.
We have used multiple measures to quantify this behavior, but generally look at the fraction of licks in the reward zone as a measure of learning (\ref{item:into:techniques:GOL:readout}).

\section{Data analysis pipeline}\label{sec:intro:techniques:pipeline}
At the time nothing else existed.

A typical 10 minute imaging session generates a $\approx$15~GB raw movie that needs to be processed down to a single spatial tuning vector for each of approximately 500 cell somas in the field of view.
Processing each movie requires identifying regions of interest (ROIs), extracting fluorescent signals from each ROI, cleaning up each trace by calculating changes in fluorescence, identifying significant calcium events, aligning calcium event times to the mouses' position in the behavioral environment, and finally calculating place fields for each cell.
Place fields from a single session can tell us how the mice represent the environment in a given session, but to address some of the more interesting questions I wanted to ask how these representations changed over time.
For this, I needed to register ROIs from day-to-day and compare the tuning of the population as a whole over time, but also track changes in the tuning of individual pyramidal cells from session-to-session.
All of these steps need to be performed fast, reliably, and as automatically as possible.
Over the course of my PhD thesis work, I, along with other members of Attila's Lab (especially Patrick Kaifosh and Nathan Danielson), developed a set of programs and tools to complete all of these steps mentioned above.
I will focus here on the pieces that I personally originated or was the primary contributor to, but "no man is an island"; Attila's lab is collaborative by design and all the code we use is shared, so multiple people have contributed to most aspects of the analysis pipeline.

\subsection{Initial processing}

\subsection{Signal extraction}

\subsection{Signal processing}
After signals are extracted for each ROI the next step is to quantify the change in fluorescence over time, as absolute fluorescent values are not interpretable in isolation.
Calculating the change in fluorescent over time is simple in theory: $$\frac{\Delta F}{F}(t) = \frac{F(t) - F_{\circ}}{F_{\circ}},$$ in practice there are many complications that any algorithm needs to be robust to.
\citep{Jia2011}

\subsection{Place cell identification}
Pyramidal cells in the mammalian hippocampus fire at specific locations within a familiar environment.
We developed an algorithm to reliably identify cells that had a significantly higher mutual information between the mouse's position and the calcium activity of the cell than expected by chance.
The overall goal of our analytical approach is to identify as un-­biased a population of spatially-­tuned cells as possible.
In particular, we aim to avoid biases in place  field width, number of place fields per cell, location of place fields on the belt, or overall cell activity/firing rate level.
In addition, unlike with electrophysiological data, calcium events have a finite duration that we needed to account for on a per-­cell basis.  

In brief, we calculate the occupancy normalized transient rate histogram for each cell with bin sizes ranging from 2~cm to 100~cm.
In addition, for each bin size we also rotate the binning through all the possible shifts, such that, for example, if a cell fired transients over any continuous 20~cm segment of the belt, there would be a 20~cm bin that contained all of the transients.

We then calculate the spatial information for each of these bin sizes and bin shifts and take the maximum value as the true spatial information of that cell.
We then shuffle all of the transients within each cell, re-­calculate the occupancy normalized transient rate histogram for each bin size and shift as described above, again take the maximum spatial information for this particular shuffle, and then create a distribution of spatial information values across shuffles.
This distribution empirically defines our spatial information confidence threshold for the particular pattern of transient durations and timings for the particular cell.

Additional details of this method can be found in \autoref{sec:df:methods:pc_identification}.
 
\subsection{Lab Analysis Bundle (LAB)}
Over the years, as the lab and our code base grew, the collection of scripts, classes, and functions used for our analysis became hard to mention and difficult for new members to learn.
In order to stabilize the code for future members, I spearheaded an effort to clean-up, refactor, and document all of the code we used in the lab.
This became the Lab Analysis Bundle (LAB), which is a complete Python package hosted privately on GitLab (\url{https://gitlab.com}).
The package is easily installable using standard Python packaging tools, with well-defined dependencies.
As part of refactoring the code, I added automatic documentation to the core functions and standardized the API.
While far from finished, the lab code is now in a much more manageable structure that has now successfully been passed on to future lab members.

\subsection{Place cell stability}\label{sec:intro:techniques:place-cells}